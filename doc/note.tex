\documentclass[11pt,a4paper]{article}
\usepackage[utf8]{inputenc}
\usepackage{amsmath, stackengine}
\usepackage{graphicx}
\usepackage{float}
\usepackage{physics}
\usepackage{mathtools}
\usepackage{listings} 
\usepackage{xcolor}
\usepackage{tikz}
\usepackage{tikz-feynman}
\usepackage{geometry}
\usepackage{hyperref}
\usepackage{listings}
\usepackage{xcolor}
\hypersetup{
   colorlinks=true,
   linkcolor=blue,
   filecolor=magenta,      
   urlcolor=cryan,
}
\urlstyle{same}
\geometry{
   a4paper,
   total={180mm, 260mm},
   left=20mm,
   top=10mm,
}
\usepackage[T1]{fontenc}

\definecolor{codegreen}{rgb}{0,0.6,0}
\definecolor{codegray}{rgb}{0.5,0.5,0.5}
\definecolor{codepurple}{rgb}{0.58,0,0.82}
\definecolor{backcolour}{rgb}{0.95,0.95,0.92}

\lstdefinestyle{mystyle}{
   backgroundcolor=\color{backcolour},
   commentstyle=\color{codegreen},
   keywordstyle=\color{magenta},
   numberstyle=\tiny\color{codegray},
   stringstyle=\color{codepurple},
   basicstyle=\ttfamily\footnotesize,
   breakatwhitespace=false,
   breaklines=true,
   captionpos=b,
   keepspaces=true,
   numbers=left,
   numbersep=5pt,
   showspaces=false,
   showstringspaces=false,
   showtabs=false,
   tabsize=2
}
\lstset{style = mystyle}

\title{Neutrino Physics Note}
\author{Tony G. Liu}

\begin{document}
    \maketitle
    \section{Quantum Kinetic Equations for Neutrino}%
    \label{sec:Neutrino Quantum Kinetics}
    
    \subsection{Wigner phase-space density operator}%
    \label{sub:Wigner phase-space density operator}
    The momentum expansion of Dirac fermionic fields (promote to operators):
    \[
    \psi(x) = \sum_{s}\int \frac{d^{3}p}{(2\pi)^{3}}\hspace{0.1cm} a^{s}_{\vb{p}}u^{s}(p)e^{-ip\cdot x} + b^{s \dagger}_{\vb{p}}v^{s}(p) e^{ip \cdot x}.\] 
    \[
    \psi^{\dagger}(x) = \sum_{s}\int \frac{d^{3}p}{(2\pi)^{3}}\hspace{0.1cm} a^{s \dagger}_{\vb{p}}u^{s \dagger}(p)e^{ip\cdot x} + b^{s}_{\vb{p}}v^{s \dagger}(p) e^{-ip \cdot x}.\] 
   
    \noindent They obey the Dirac equation: $(i\gamma ^{\mu} \partial _{\mu} - m) \psi(x) = 0$\\

    \noindent A Fourier transformation connects momentum and position space representation of the operators:
    \begin{align*}
        a^{s}_{\vb{p}} &= \int d^{3}x \hspace{0.1cm} e^{ip \cdot x}u^{s \dagger}(p) \psi(x)   \hspace{0.5cm} a^{s \dagger}_{\vb{p}} = \int d^{3}x \hspace{0.1cm} e^{-ip \cdot x} \psi^{\dagger}(x) u^{s}(p)\\
        b^{s}_{\vb{p}} &= \int d^{3}x \hspace{0.1cm} e^{ip \cdot x} \psi ^{\dagger}(x)v^{s}(p)  \hspace{0.5cm} b^{s \dagger}_{\vb{p}} = \int d^{3}x \hspace{0.1cm} e^{-ip \cdot x} v^{s \dagger}(p) \psi(x)\\
        \text{using the normalization: } & u^{s\dagger}(p)u^{s^{\prime}}(p) = v^{s \dagger}(p)v^{s^{\prime}}(p) = \delta_{ss^{\prime}}.
    \end{align*}
*Note: The normalization can be up to some factors like $2E_{\vb{p}}$ or $\frac{E_{\vb{p}}}{m}$. It all depends on which factor is more convenient for establishing the theory.\\

    \noindent According to the spin-statistics theorem, at equal time the Pauli exclusion principle is implemented by anti-commutation relations of the field operators:
    \[
    \comm{\psi(\vb{x}, t)}{\psi(\vb{x}^{\prime}, t)} = \delta(\vb{x} - \vb{x}^{\prime}) .\] 
    \noindent The creation and annihilation operators then automatically satisfy the relation:
    \[
       \acomm{a_{\vb{p}}^{s}}{a_{\vb{p}^{\prime}}^{s^{\prime} \dagger}} = \acomm{b_{\vb{p}}^{s}}{b_{\vb{p}^{\prime}}^{s^{\prime}\dagger}} = (2\pi)^{3} \delta_{s s^{\prime}}\delta^{3}(\vb{p} - \vb{p}^{\prime}) 
 .\]
 \textbf{Proof}:
    \begin{align*}
       \acomm{a_{\vb{p}}^{s}}{a_{\vb{p}^{\prime}}^{s^{\prime} \dagger}} &= \int d^{3}x d^{3}x^{\prime} \hspace{0.1cm}e^{i(p \cdot x - p^{\prime}\cdot x^{\prime})} u^{s \dagger}(p) u^{s^{\prime}}(p^{\prime})\acomm{\psi (\vb{x}, t)}{\psi ^{\dagger}(\vb{x}^{\prime}, t)} = \int d^{3}x e^{i(p-p^{\prime}) \cdot x} u^{s \dagger}(p) u^{s^{\prime}}(p^{\prime}) \\
            &\text{Similarly for the antifermion operators.}
    \end{align*}

    \noindent Now, consider the left-handed massless neutrino field (with spin $s$ = $\frac{1}{2}$):
\begin{align*}
   \psi(x) &= \int \frac{d^{3}p}{(2\pi)^{3}}\Big(a_{\vb{p}}(t)u_{\vb{p}} + b^{\dagger}_{-\vb{p}}(t)v_{-\vb{p}}\Big)e^{i \vb{p} \cdot \vb{x}} \hspace{0.5cm}\\ 
   \psi ^{\dagger}(x) &= \int \frac{d^{3}p}{(2\pi)^{3}}\Big(a_{\vb{p}}^{\dagger}(t)u_{\vb{p}}^{\dagger} + b_{-\vb{p}}(t)v^{\dagger}_{-\vb{p}}\Big)e^{-i \vb{p} \cdot \vb{x}}\\
   u_{\vb{p}}&: \text{negative-helicity fermionic spinor}\\
   v_{\vb{p}}&: \text{positive-helicity fermionic spinor}
 \end{align*}


 \section{Numerical Specification}%
 \label{sec:Numerical Simulation}

 The e.o.m. for the mixed state of neutrinos described by mean-field density matrix $\varrho(t,\vb{x}, \vb{p})$:
 \[
 \left(  \frac{\partial}{\partial t} + \vb{v}\cdot \frac{\partial}{\partial \vb{x}} + \vb{f} \cdot \frac{\partial}{\partial \vb{p}}\right) \varrho(t,\vb{x}, \vb{p}) = i \comm{\varrho (t, \vb{x}, \vb{p})}{\mathcal{H}(t, \vb{x}, \vb{p})} + C[\varrho].\] 

 \begin{align*}
    \varrho(t,\vb{x}, \vb{p}) &= \frac{f_{\nu_{e}} + f_{\nu_{x}}}{2}\mathcal{I} + G_{\nu}(\vb{p}) \rho(t,\vb{x}, \hat{\vb{p}})\\
    \mathcal{H}_{\nu\nu} &= \sqrt{2}G_{F}\int \frac{d^{3}\vb{q}}{(2\pi)^{3}}(1 - \hat{\vb{p}}\cdot \hat{\vb{q}})\big(\varrho_{\vb{q}}(t, \vb{x}) - \bar{\varrho}_{\vb{q}}(t, \vb{x})\big)
 .\end{align*}

 \noindent We may integrate out the energy $\abs{\vb{q}} = \varepsilon_{\nu}$, and define the angular ELN distribution as: 
 \[
 g_{\nu}(\hat{\vb{p}}) = \frac{1}{n_{\nu_{e}}} \int \frac{\varepsilon_{\nu}^{2}d\varepsilon_{\nu}}{2\pi^{2}}\big(G_{\nu}(\vb{p}) - G_{\bar{\nu}}(\vb{p})\big) .\]
 So that the interaction Hamiltonian becomes:
 \[
 H_{\nu\nu}(t, \vb{x}, \hat{\vb{p}}) = \mu \int \frac{d \hat{\vb{q}}}{4\pi}(1 - \hat{\vb{p}} \cdot \hat{\vb{q}}) \Big( g_{\nu)}(\hat{\vb{q}}) \rho_{\hat{\vb{q}}} - g_{\bar{\nu}}(\hat{\vb{q}}) \bar{\rho}_{\hat{\vb{q}}} \Big) ,  \hspace{0.1cm}\mu = \sqrt{2}G_{F}n_{\nu_{e}} .\] 

\noindent Given the explicit form:
\begin{align*}
   \hat{\vb{p}} &= \vb{v} = (\sqrt{1-v_{z}^{2}}\cos \varphi,  \sqrt{1-v_{z}^{2}}\sin \varphi, v_{z})\\ 
   \int d\hat{\vb{q}} &= \int _{-1}^{1}dv_{z}\int_{0}^{2\pi} d\varphi\\
   \vb{v} \cdot \nabla &= \sqrt{1-v_{z}^{2}} \Big(\cos \varphi \frac{\partial}{\partial x} + \sin\varphi \frac{\partial}{\partial y} \Big) + v_{z} \frac{\partial}{\partial z} 
.\end{align*}



\subsection{Time Independent Hamiltonain}%
\label{sub:Time Independent Hamiltonain}
\begin{align*}
   \left(  \frac{\partial}{\partial t} + \vb{v}\cdot \frac{\partial}{\partial \vb{x}} \right) \varrho(t,\vb{x}, \vb{p}) &= i \comm{\varrho (t, \vb{x}, \vb{p})}{\mathcal{H}(t, \vb{x}, \vb{p})} , \hspace{0.1cm}\varrho(t=0,\vb{x}) = f(\vb{x})\\
   \mathcal{H} &= \begin{pmatrix}
 -\cos 2\theta & \sin 2\theta\\
\sin 2\theta & \cos 2\theta \\
\end{pmatrix}\\
      \Rightarrow \varrho(t,\vb{x}) &= e^{-i\mathcal{H}t}f(\vb{x} - \vb{v}t)e^{i\mathcal{H}t} 
.\end{align*}



Given the explicit form of initial condition, the solution is
\begin{align*}
   e^{i\mathcal{H}t} &= \begin{pmatrix} 
      e^{it}\sin^{2}{\theta} + e^{-it}\cos^{2}{\theta}& i\sin{t} \sin{2\theta}\\
      i\sin t  \sin 2\theta & e^{it} \cos^{2}\theta + e^{-it}\sin^{2}\theta\\
   \end{pmatrix} , 
   f(\vb{x}) = \begin{pmatrix} 
      f_{ee}(\vb{x}) & 0\\
      0 & f_{xx}(\vb{x})\\
   \end{pmatrix}\\
   \\
      \varrho_{ee}(t,\vb{x}) &=  
      f_{ee}(\vb{x} - \vb{v}t) \Big(\cos^{4}(\theta) + 2\cos (2t) \cos ^{2} (\theta) \sin^{2}(\theta) + \sin^{4} (\theta)\Big) + f_{x x}(\vb{x} - \vb{v}t)  \sin ^{2}t \sin ^{2 }2\theta \\ 
      \varrho_{ex}(t, \vb{x}) &= \varrho_{xe}^{*} = 2i \Big(f_{ee}(\vb{x} - \vb{v}t) - f_{x x} (\vb{x} - \vb{v}t)\Big) \hspace{0.1cm}\sin(t) \sin(\theta) \cos(\theta) \Big(e^{it}\cos^{2}(\theta) + e^{-it}\sin ^{2}(\theta)\Big)\\
      \varrho_{x x}(t, \vb{x}) &= f_{x x}(\vb{x} - \vb{v}t)\Big(\cos^{4}(\theta) + 2 \cos(2t) \cos^{2}(\theta) \sin^{2}(\theta)  + \sin ^{4}(\theta) \Big)  + f_{ee}(\vb{x} - \vb{v}t)\sin ^{2}(t) \sin ^{2}(2\theta)
.\end{align*}
\\


\subsection{Numerical Approach in 1+2+2 dimensions}%
\label{sub:Numrical Approach in 1+2+2 dimensions}

\begin{itemize}
   \item $\partial _{t} \varrho = - \vb{v}\cdot \nabla \varrho + i \comm{\varrho}{\matcal{H}} \equiv g(t,  \varrho)$
   \item $\vb{v} \cdot \nabla \varrho \rightarrow \frac{v_{x}}{12 dx}\left(\varrho_{i-2,j} - 8\varrho_{i-1, j} + 8\varrho_{i+1, j} - \varrho_{i+2,j }\right) + \frac{v_{z}}{12dz} \left(  \varrho_{i,j-2} -8\varrho_{i,j-1} + 8\varrho_{i,j+1} -\varrho_{i,j+2} \right)$
   \item index : $(j+g_{z})*(N_{x} + 2*g_{x}) + (i+g_{x})$
   \begin{itemize}
      \item $i = 0 \sim  N_{x}-1, \hspace{0.1cm} j = 0 \sim N_{y}-1$
   \end{itemize}
   \item $g_{x}, g_{z}$ are number of grids in ghost zones
   \item RK4 -- $\varrho ^{n+1} = \varrho^{n} + \frac{dt}{6}\left( g_{0} + 2 g_{1} + 2g_{2} + g_{3}\right) + O(dt ^{5})$
        \begin{itemize}
           \item  $g_{0} = g(t_{n}, \varrho^{n})$
           \item $g_{1} = g(t_{n} + \frac{dt}{2}, \varrho^{n} + \frac{dt}{2}g_{0})$
           \item $g_{2} = g(t_{n} + \frac{dt}{2} , \varrho^{n} + \frac{dt}{2}g_{1})$
           \item $g_{3} = g(t_{n} + dt, \varrho^{n} + dt g_{2})$
        \end{itemize}
 \end{itemize}



\subsection{Numerical Setup}%
 \label{sub:Perturbation from an initial state}

 Consider two-flavor system, and impose the translation symmetry on both $x$ and $y$ dimensions, the e.o.m. for neutrino and antineutrino: 
\begin{equation*}
   \left(   \partial _{t} + v_{z}\partial_{z}\right)\rho(t,z,v_{z}) = i \comm{\rho(t,z,v_{z})}{\mathcal{H}(t,z,v_{z})}
\end{equation*}

\begin{equation*}
   \left(   \partial _{t} + v_{z}\partial_{z}\right)\overline{\rho}(t,z,v_{z}) = i \comm{\overline{\rho}(t,z,v_{z})}{\overline{\mathcal{H}}(t,z,v_{z})}
\end{equation*}

\begin{equation*}
   \text{with } \rho(t,z,v_{z}) = \begin{pmatrix}
      \rho_{ee} & \rho_{ex}\\
      \rho^{*}_{ex} & \rho_{x x}\\
   \end{pmatrix} \text{, }
   \overline{\rho}(t,z,v_{z}) = \begin{pmatrix}
      \bar\rho_{ee} & \bar\rho_{ex}\\
      \bar\rho^{*}_{ex} & \bar\rho_{x x}\\
   \end{pmatrix}
\end{equation*}

\noindent Ignoring the MSW effect, the Hamiltonian would be: 
\begin{equation*}
   \mathcal{H}(t,z;v_z)=
\begin{pmatrix}
 -\cos 2\theta & \sin 2\theta\\
\sin 2\theta & \cos 2\theta \\
\end{pmatrix}
+
\mu \int_0^1 dv_z' (1-v_{z} v_z')
[\rho(t,z;v_z')-\bar\rho^*(t,z;v_z')]
\equiv \mathcal{H}_{vac} + \mathcal{H}_{\nu\nu},
\end{equation*}

\begin{equation*}
   \overline{\mathcal{H}}(t,z;v_z)=
\begin{pmatrix}
   -\cos 2\theta & \sin 2\theta\\
    \sin 2\theta & \cos 2\theta \\
\end{pmatrix}
-
\mu \int_0^1 dv_z' (1-v_{z} v_z')
[\rho^*(t,z;v_z')-\bar\rho(t,z;v_z')]
\equiv \overline{\mathcal{H}}_{vac} + \overline{\mathcal{H}}_{\nu\nu}
\end{equation*}

\noindent The finite difference index:

\begin{center}
   \begin{tabular}{c c c c}
      & $t$ & $z$ & $v_{z}$ \\
      step & $\delta t$ &  $\delta z$ &  $\delta v_{z}$\\
      numbers& $N_{t}$ & $N_{z}$ & $N_{v_{z}}$ \\
      index & $i_1$ & $i_2$ & $k$
   \end{tabular}
\end{center}

\noindent Let $\rho(t,z, v_{z})$ be $\rho^{k}_{i_1, i_2}$ in the discretization and the interaction Hamiltonian\\
\[
\mathcal{H}^{k}_{\nu\nu, i_1, i_2}  = \frac{\mu}{N} \hspace{0.1cm}\sum_{k'=0}^{N} \Big(1 - \frac{kk'}{N^{2}}\Big) \left( \rho^{k'}_{i_1, i_2} - \bar\rho^{*k'}_{i_1, i_2}\right).\] 
\[
\overline{\mathcal{H}}^{k}_{\nu\nu, i_1, i_2}  = -\frac{\mu}{N} \hspace{0.1cm}\sum_{k'=0}^{N} \Big(1 - \frac{kk'}{N^{2}}\Big) \left( \rho^{*k'}_{i_1, i_2} - \bar\rho^{k'}_{i_1, i_2}\right).\] 



\noindent The above Courant-Friedrichs-Lewy (CFL) stability criterion:
\[
c \equiv \frac{\abs{v_{z}}\delta t}{\delta z}\leq 1 .\] 



\end{document}
